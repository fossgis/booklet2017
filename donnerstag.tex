\renewcommand{\konferenztag}{\donnerstag}
% 10:30
\newsmalltimeslot{09:30}
\abstractNeun{Roman Zoller}%
{Ngeo: OpenLayers meets Angular}%
{}%
{Ngeo ist eine Open-Source-JavaScript-Bibliothek, die eine Kombination der Funktionalität von
OpenLayers und der Modularität von AngularJS ermöglicht. Sie stellt AngularJS-Services und
Komponenten zur Verfügung, die als Bausteine für GIS-Webanwendungen benutzt werden können. Dieser
Vortrag zeigt anhand von konkreten Codebeispielen auf, wie Ngeo die Softwareentwicklung vereinfacht.
Wir beschreiben, wie Ngeo in moderne Applikationen integriert werden kann, und geben einen Einblick
in unsere Strategie, um Ngeo im sich schnell entwickelnden JavaScript-Umfeld auf dem Laufenden zu
halten.}


\abstractElf{Armin Retterath}%
{INSPIRE vs. Open Data? -- Probleme und mögliche Lösungen}%
{}%
{Im Vortrag werden die Unterschiede und Gemeinsamkeiten von GDI- und Open-Data-Architekturen anhand
praktischer Beispiele herausgearbeitet. Auf der einen Seite steht ein starrer, auf Normen
basierender Ansatz, auf der anderen Seite ist die Enwicklung eher Community getrieben und die
Standardisierung steht noch aus.

Aufgrund des großen politischen Drucks in Richtung Open Data steigt
auch der Druck auf Geodatenprovider ihre Metadaten Open-Data-kompatibel bereitzustellen. Es gibt
hier einige semantische und technische Hürden, die man kennen und überwinden muss. Der Beitrag gibt
hierzu Hilfestellung.}

\abstractDreizehn{Roland Olbricht}%
{Habe Kante, suche Route}%
{}%
{Auf OpenStreetMap-Daten lässt sich für viele Verkehrsmittel routen. Davon weiß man aber nicht,
  welche Rolle eine einzelne Kante für alle potentiellen Routen spielt.

In diesem Vortrag wird eine Kenngröße ermittelt, die einer Kante ihre mathematische Bedeutung im
Netzwerk zuordnet. Dies zeigt Kanten auf, deren Bedeutung stark von ihrer a-priori-Wichtigkeit
abweicht, z.\,B. Autobahnausfahrten, die sich gegen die durchgehende Autobahn durchsetzen.

Es wird in diesem Vortrag die Kenngröße exakt definiert, eingeordnet
und es werden interessante Kanten im OSM"=Netzwerk untersucht.
}

\newtimeslot{10:00}
\abstractNeun{Oliver Jeker}%
{Kennst du deine GDI?}%
{}%
{Als Verantwortliche einer GDI sind wir regelmäßig mit typischen Fragen konfrontiert die wir heute
  nicht ohne Stirnrunzeln beantworten können. Wo wirkt sich eine Schemaänderung des
  Parzellendatensatzes in der GDI überall aus? Wieso läuft das QGIS-Plug-In des ehemaligen
  Arbeitskollegen nicht mehr?

Im Rahmen der Erneuerung der GDI des Kantons Solothurn wurde ein GDI-Datenmodell erstellt, welches
die Zusammenhänge zwischen Datensätzen, Diensten und Anwendungen abbildet. In der Präsentation wird
das Datenmodell vorgestellt und wie es die einleitenden Fragen beantwortet.}

\abstractElf{Axel Schaefer}%
{Neues in Metador}%
{Make Metadata Great Again}%
{Der quelloffene Metadateneditor MetaDor wird zur Zeit als einfach zu bedienender und schnell
anzupassender Editor für Metadaten genutzt. Die aktuellen Entwicklungen erweitern ihn nun um einen
CSW-Publisher, sodass die Software selbst die Metadaten über eine CSW"=Schnittstelle nach draußen
publizieren kann. Zudem sind die Standardformulare für Geodaten und Dienste mit den
Konventionen des GDI-DE-Arbeitskreises erweitert worden. Der Vortrag stellt die spannenden
Neuerungen in einer aktuellen Entwicklerversion vor.}

\abstractDreizehn{Robert Klemm}%
{Thesis GraphHopper-Routing mit Maut-Erweiterung}%
{Einen Versuch die OSM-LKW-Mautdaten in der GraphHopper-Direction-API zu integrieren}%
{Im Rahmen der Masterarbeit ging es um einen Entwurf einer erweiterten
Routinglösung, die von der Firma GraphHopper GmbH als
Basis-Routinglösung zur Verfügung gestellt wurde. Die Berechnung der
Route muss neben der Wegstrecke zusätzlich die Fahrzeugklasse und deren
Kosten für eine Infrastrukturabgabe berücksichtigen (Maut-Routing).
Ziel war es, die frei verfügbaren Mautinformationen aus verschiedenen
Quellen zu verwenden, um ein offlinefähiges Routing unter Einbeziehung
der LKW-Maut oder anderer LKW-Maut-relevante Attributinformationen als
Routingprofil in GraphHopper zu erlauben.}

\newtimeslot{10:30}
\abstractNeun{Arne Schubert}%
{Mobile Nutzung von Geodaten mit \mbox{einem} Leaflet-basierten Offline-Client}%
{}%
{In einer technischen Machbarkeitsstudie hat die WhereGroup eine hybride mobile
Anwendung erstellt, die Daten sowohl online beziehen kann als auch offline
vorhält. Durch die hybride und modulare Struktur kann die Anwendung flexibel
auf den benötigten Funktionsumfang angepasst werden, als auch auf allen
typischen mobilen Geräten benutzt werden.

Dabei setzt die Anwendung auf
bewährte moderne Frameworks wie Ionic inkl. Cordova sowie YAGA leaflet-ng,
welches Leaflet in Angular integriert. Die Offlinedaten können als Geopackage
importiert werden.}

\abstractElf{Dirk Stenger}%
{Erhöhe den Nutzen deines Dienstes}%
{Qualitätskontrolle für OGC-konforme Geodatendienste mit TEAM Engine}%
{Um die Implementierung von GIS-Software zu unterstützen, stellt das Open Geospatial Consortium
(OGC) mehrere Validatoren basierend auf der TEAM Engine zur Verfügung. Die TEAM Engine ist ein
Testframework, mit dem Entwickler Geodatendienste, wie zum Beispiel WFS und WMS, testen können.  In
diesem Vortrag wird gezeigt, wie die TEAM Engine zum Erstellen einer WFS~2.0"=Referenzimplementierung
verwendet wurde. Zudem erfolgt eine Demonstration, wie die TEAM Engine und ein vollständiger deegree
WFS~2.0 inklusive PostgreSQL-Datenbank mit Docker in einer Entwicklungsumgebung eingesetzt werden
können.}

\abstractDreizehn{Frederik Ramm}%
{Routing Engines für OpenStreetMap}%
{Eine Marktübersicht}%
{Inzwischen buhlen eine Reihe auf OpenStreetMap spezialisierter Open-Source-Routing-Engines um
Aufmerksamkeit. Neben den Platzhirschen OSRM und Graphhopper gibt es \emph{Valhalla} von Mapzen, den
Newcomer \emph{Itinero}, ältere Programme wie \emph{routino} und Speziallösungen wie \emph{brouter} für das
Fahrradrouting. Dieser Vortrag möchte die Stärken und Schwächen der verschiedenen Lösungen aufzeigen
und Empfehlungen für verschiedenen Anwendungsbereiche aussprechen.}
\enlargethispage{0.2\baselineskip}
\vspace{-0.5\baselineskip}
\sponsorenboxA{209_gbd-consult}{0.27\textwidth}{2}{%
\textbf{Bronzesponsor}\\
Die Geoinformatikbüro Dassau GmbH bietet Lösungen für die Umsetzung 
von GIS"=Projekten auf Open"=Source"=Basis und ist spezialisiert auf die 
Software GRASS und QGIS.
}



\newtimeslot{11:30}

\abstractNeun{Christian Mayer}%
{GeoExt 3 in der Praxis}%
{}%
{GeoExt ist eine auf den JavaScript-Bibliotheken Open\-Lay\-ers und ExtJS aufbauende
  Open"=Source"=JavaScript"=Bibliothek, die es vereinfacht, Kartenmaterial in ansprechenden und komplexen
  Oberflächen zu präsentieren.  Der Vortrag stellt zunächst die Neuerungen von GeoExt 3 sowie die
  Neuerungen der Vater-Bibliotheken
vor.  Anschließend werden Projekte aus der Praxis gezeigt, bei denen GeoExt~3 als zentrales
Webmapping-Framework eingesetzt wird. Dabei werden die vielfältigen Einsatzmöglichkeiten zum Aufbau
von WebGIS mit GeoExt gezeigt sowie nützliche Tipps aus der Praxis und "`Dos and Don'ts"'
vorgestellt.}

\abstractElf{Sebastian Goerke}%
{Der GIS-Arbeitsplatz der Zukunft}%
{}%
{Mobilität gewinnt bei der Gestaltung des Arbeitslebens immer mehr an Bedeutung. Arbeitnehmern ist
es zunehmend wichtig, selbst über die Gestaltung des Arbeitstages zu bestimmen und damit
einhergehend auch ortsunabhängig arbeiten zu können. Diese Entwicklungen betreffen auch die
Geobranche. Wie soll er also aussehen, der GIS-Arbeitsplatz der Zukunft?  Die Auslagerung von
Anwendungen in die Cloud ist geeignet, Einsparungen im IT-Betrieb herbeizuführen. Aber ein typischer
GIS"=Arbeitsplatz besteht heute aus Desktop-Anwendungen. Desktop-Anwendungen in der Cloud? Eine
klassische Antwort lautet hier Web-GIS.}

\abstractDreizehn{Tim Alder}%
{Geodaten in der Wikipedia}%
{Ein Update 2017}%
{Der Vortrag soll den aktuellen Stand für die Geodaten- und Kartennutzung im Bereich der Wikipedia
wiedergeben. Teilaspekte sind Geodaten und deren Visualisierung in Wikidata, ein graphischer
Karteneditor für die Wikipedia sowie die Speicherung von Geometriedaten in Commons.  }

%HACK
\vspace{0.7\baselineskip}
\noindent\sponsorenboxA{201-omniscale}{0.38\textwidth}{2}{%
\textbf{Bronzesponsor}\\
Omniscale ist Spezialist für das Bereitstellen von schnellen und attraktiven
Karten auf Basis von OpenStreetMap und amtlichen Daten. Unter
maps.omniscale.com bietet Omniscale kostengünstige Kartendienste für Desktop-
und Online-Anwendungen an. Omniscale ist zudem Initiator von MapProxy und
Imposm.}


\newtimeslot{12:00}
\abstractNeun{Markus Neteler}%
{GRASS GIS -- Projektstatus und Neuerungen der Version 7.2}%
{}%
{Nach mehr als zwei Jahren Arbeit ist ein neues Major Release von GRASS GIS, die Version 7.2.0,
erschienen. Neben Performance- und Effizienz-Verbesserungen enthält die neue Version eine ganze
Reihe an innovativen Modulen zur Analyse von Raster- und Vektordaten sowie zur Zeitreihenanalyse,
darunter einen visuellen Datenkatalog, LiDAR"=Unterstützung, einen integrierten Python"=Editor,
Berechnung von 3D"=Fließakkumulation und 3D"=Gradienten
und neue effiziente Rasterdatenkompression.

Der Vortrag erläutert den aktuellen Stand des GRASS-GIS-Projekts, geht insbesondere auf die
Neuerungen der Version 7.2 ein und stellt einige der neuen Module exemplarisch vor.
}

\abstractElf{Jörg Höttges}%
{QKan -- Kanalkataster mit QGIS}%
{Projektbeschreibung und aktueller Entwicklungsstand}%
{Basierend auf QGIS werden eine Datenbankstruktur sowie verschiedene Plugins für ein Kanalkataster
erstellt, mit der die Kanaldaten vor allem für Berechnungsprogramme vorbereitet und die
Berechnungsergebnisse aufbereitet und weiterverwendet werden können. Als Datenbanken sind SpatiaLite
und PostGIS vorgesehen. Die Programmierung erfolgt mit Python-basierten Plugins. Als anzubindende
Simulationsprogramme sind zunächst HYSTEM-EXTRAN (ITWH), Kanal++ (tandler.com) sowie SWMM geplant.}

\abstractDreizehn{Arndt Brenschede}%
{OSM-Datenformate für Anwendungen}%
{Der Weg zu verlustfreien Vektortiles}%
{OSM-Daten werden bis heute üblicherweise in speziellen Datenformaten für spezielle Anwendungsfälle
aufbereitet, was für Einsteiger unübersichtlich ist und dem Anwender die Handhabung erschwert.
Dieser Vortrag gibt einen Überblick über die gängigen Lösungen mit ihren Vor- und Nachteilen sowie
einen Ausblick auf die Möglichkeit, OSM-Daten ohne Informationsverlust in einem universellen,
indexierten und gekachelten Format darzustellen~-- verlustfreie Vektortiles.}

\newtimeslot{12:30}
\abstractNeun{Otto Dassau}%
{Neues von QGIS}%
{}%
{Das QGIS-Projekt ist sicherlich eines der aktivsten Open-Source-GIS-Projekte.
Das bezieht sich einerseits auf die Entwicklung der Software,
andererseits auf das Projekt und seine Community selbst. Es haben sich eine
Vielzahl von Veränderungen ergeben und eine weitere Anzahl von Neuerungen
stehen in den kommenden Monaten an. Der Vortrag gibt einen Überblick zu folgenden Themen:
\begin{itemize}\setlength\itemsep{-1pt}
\item Das internationale QGIS-Projekt ist jetzt eine Association (Verein).
\item Der deutsche Verein QGIS Anwendergruppe Deutschland e.V.
\item Die Software QGIS macht einen Versionssprung, was verbirgt sich dahinter?
\end{itemize}%
}

\abstractElf{Till Adams}%
{The real Big Data}
{Potentiale eines satellitenbildgestützten Temperaturdatenarchivs}%
{Im Rahmen eines Forschungs- und Entwicklungsprojektes bauen wir (Mundialis) ein 15 Jahre
  zurückreichendes Temperaturdatenarchiv für ganz Europa, das auf Satellitendaten basiert, auf.
  Das Archiv beinhaltet vier Temperaturschritte pro Tag und hat eine räumliche Auflösung von einem
  Kilometer und ist damit genauer und besser, als jegliche, auf Interpolation von wenigen
  Klimastationen basierenden interpolierte Datensätze.
  Der Vortrag konzentriert sich auf zwei Hauptaspekte: die
  Verarbeitung von Fernerkundungsdaten im Terabyte-Bereich mit High-Performance-Computing mit
  GRASS GIS und potentielle Anwendungsfelder der aus diesem Archiv generierten Information}

\abstractDreizehn{Roman Härdi}%
{Feuerwehreinsatzkarten mit OSM}%
{Wenn es brennt, eilt es!}%
{Karten sollen für die
  Anfahrt und für den Einsatz die notwendigen Informationen gut lesbar zur Verfügung stellen.
Der Vortrag zeigt die für die Feuerwehr wichtigen Informationen mit Geobezug.

Welche gibt es schon
in OSM? Wie können aus der Reihe der Feuerwehr neue Mapper geworben, motiviert und unterstützt
werden?  Es wird der Prozess von der Datenerfassung, Rendering bis zur Erzeugung des Ausschnitts für
die
Anfahrt und der Einsatz anhand eines Beispiels gezeigt. Die offenen Herausforderungen sind fehlende
Tags und das Schützen der kritischen Nodes vor falscher Bearbeitungen.}

\newtimeslot{14:00}
\abstractNeun{Claas Leiner}%
{QGIS maßgeschneidert!}%
{QGIS mit individueller Konfiguration und Modellerwerkzeugen für unerfahrene Nutzer handhabbar machen}%
{QGIS ist zusammen ist inzwischen eine sehr mächtige und komplexe Software geworden, die unerfahrene
  Anwender überfordern kann.  Häufig soll QGIS in Büros oder Behörden für ganz spezielle Aufgaben
  von Mitarbeitern ohne besondere
GIS-Kenntnisse angewendet werden.  Mit einem individuellen Konfigurationsverzeichnis und
Modellerwerkzeugen lässt sich die QGIS-Benutzeroberfläche so vereinfachen, dass auch unerfahrene
Nutzer erfolgreich mit ihren Daten arbeiten können. 
}

\abstractElf{Volker Grescho, Roland Krämer}%
{Open Source, Open Data und Citizen Science in der Biodiversitätsforschung}%
{Wie Wissenschaft und Verwaltung von freiem Zugang und Bürgerbeteiligung profitieren}%
{Weltweit schreitet der Verlust von Artenvielfalt und intakten Lebensräumen mit ernstzunehmender
Geschwindigkeit voran. In Deutschland existiert eine Vielzahl von Projekten, die Daten zu Natur und
Umwelt erfassen. Insgesamt werden etwa 95 Prozent dieser Daten von Ehrenamtlichen erfasst.  Durch
den Einsatz von Open-Source-Lösungen bei der Erfassung von Biodiversitätsdaten und durch Open Access
gab es in den letzten Jahren einen enormen Zuwachs an neuen Daten und an Möglichkeiten ihrer
Auswertung. Wir stellen Projekte aus dem Biodiversitätsbereich vor, die erfolgreich Open Source
und/oder Open Access sind.}

\abstractDreizehn{Michael Glanznig}%
{Meine eigene Karte: Überblick über Rendering-Techniken und Software}%
{}%
{Wer heute seine eigene Karte rendern möchte, hat soviel Auswahl wie nie. Von Raster- über
Vektortiles, von Mapnik über OSM2Vectortiles zu Mapbox. TileMill, Kosmtik oder Mapbox Studio? Ich
versuche einen Überblick über die derzeit vorhandene Renderingtechnik und -software, über ihre
Stärken und Schwächen, zu geben. Beispiele sollen überblicksmäßig verdeutlichen welche Komponenten
notwendig sind und wie sie zusammenspielen. Vektortiles sind der heißeste Kandidat für die nächste
Generation der Hauptkarte von OpenStreetMap. Ich versuche darzustellen wie hier eine mögliche
Zukunft aussehen könnte.}


\newtimeslot{14:30}
\abstractNeun{Marco Hugentobler}%
{QGIS Server Projektstatus}%
{}%
{Dieser Beitrag gibt einen Überblick über diverse Aktivitäten, die rund um QGIS
Server im Gange sind. So gibt es im Rahmen von QGIS~3 diverse Änderungen in der
Codebasis. Ende letzten Jahres wurden Änderungen eingepflegt, welche QGIS
Server zu einer WMS~1.3-konformen Software machen. Im Vortrag wird auch
gezeigt, was es für den OGC-Test braucht und was im Server geändert werden
musste, damit der Test erfolgreich durchläuft.}

\abstractElf{Pirmin Kalberer}%
{Von WMS zu WMTS zu Vektortiles}%
{Eigene Daten als Mapbox Vector Tiles publizieren}%
{Vektortiles haben das Potential die bewährten
Rasterkarten in vielen Bereichen abzulösen oder mindestens maßgeblich zu ergänzen.
Open-Source-Produkte geben dabei den Takt an und es steht bereits eine beachtliche Zahl an
Vektortile-Servern zur Verfügung, um eigene Daten in diesem effizienten Format zu publizieren. Der
Vortrag gibt einen aktuellen Überblick über die Open-Source-Lösungen zur Publikation von Mapbox
Vektor Tiles und vergleicht die Vor- und Nachteile der verschiedenen Produkte. }

\abstractDreizehn{Bernhard Fischer}%
{Smrender}%
{Ein modularer, flexibler Papierseekarten-Renderer}%
{Seekarten werden in der Berufs- und Sportschifffahrt für die Navigation von
Schiffen verwendet und haben aufgrund ihrer Sicherheitsrelevanz einen enormen
Qualitätsanspruch.

Seekarten weisen eine hohe Informationsdichte auf und müssen übersichtlich und
eindeutig verständlich sein. Das wird durch entsprechend intelligente
Platzierung und Rotation von Objekten, sowie die Wahl verschiedener Farben und
Schriftarten erreicht.

Auf den ersten Blick erscheint das nicht weiter anspruchsvoll, untersucht man
eine Seekarte jedoch im Detail, so stellen die Karteneigenschaften eine hohe
Komplexität in Hinblick auf die Entwicklung von Algorithmen dar.
Offizielle Seekarten werden computergestützt gerendert und manuell
nachbearbeitet, um eine optimale Darstellung zu erreichen.
Das Open-Source-Projekt Smrender versucht die Besonderheiten von Papierseekarten
zu implementieren.}

\newtimeslot{15:00}
\abstractNeun{Andreas Neumann}%
{QGIS Web Client 2}%
{Die zweite Generation des Webclients optimiert für QGIS Server}%
{QGIS Web Client~2 (QWC~2) ist die zweite Generation des QGIS-Webclients,
optimiert für QGIS Server. QWC~2 setzt responsive Design ein. Die
identische Version läuft auf Tablets, Mobiltelefonen und
Desktop-Rechnern. Er unterstützt die Erweiterungen des QGIS Servers für
das PDF-Drucken, Suche, Datenexport, Legenden, etc. Das Projekt kann
unter github.com/qgis/qwc2-demo-app heruntergeladen werden.
Gegenüber der ersten Generation kommen neue Bibliotheksversionen zum
Einsatz: OpenLayers~3, ReactJS, nodejs und yarn.}

\abstractElf{Oliver Tonnhofer}%
{PostGIS in Action}%
{Optimierte Datenbanken für Online-Stadtpläne}%
{Der Vortrag ist ein Erfahrungsbericht von der Umsetzung verschiedener Onlinekarten-Projekte mit
  PostgreSQL und PostGIS auf Basis von amtlichen Daten (ALK/ALKIS) und OpenStreetMap. 

Es werden die folgenden Fragen beantwortet: Wie können Daten importiert werden? Wie können Daten
strukturiert werden? Wie können Daten homogenisiert werden? Wie können Daten im laufenden Betrieb
aktualisiert werden? Wie können Daten optimiert werden?}

\abstractDreizehn{}{Lightning Talks II}{}%
{%
  \vspace{-2em}
  \begin{itemize}
    \RaggedRight
    \setlength{\itemsep}{-0.25\baselineskip} % Aufzählungspunktabstand auf 0
    \item \emph{Alexander Zipf:} DeepVGI
    \item \emph{Kolossos:} Einsatz von Lowcost-Lidarsystemen für die OSM-Landvermessung
    \item \emph{Hans Jörg Stark:} Unkonventionelle Karten mit g2jascii
  \end{itemize}
}

\sponsorenboxA{203-CS-Gis}{0.34\textwidth}{3}{%
\textbf{Bronzesponsor}\\
CSGIS hat sich auf die Entwicklung von individuellen Fachanwendungen mit
Desktop-GIS und Web-GIS spezialisiert (von kleineren Viewers bis hin zu
fortgeschrittenen WebGIS-Systemen). Wir programmieren neue Funktionen Ihres
Open-Source-Desktop-GIS oder Ihrer Web-GIS-Anwendung und entwickeln mit Ihnen
zusammen GIS-Software-Lösungen. Kunden aus verschiedenen Branchen
(Stadtverwaltung, Archäologie, Landwirtschaft, Forstwirtschaft, Medizin, etc.)
arbeiten täglich mit GIS und nutzen von uns entwickelte Fachanwendungen.%%
}


\newtimeslot{16:00}
\abstractNeun{Dirk Stenger}%
{eGovernment in der Bauleitplanung mit der xPlanBox leicht gemacht}%
{}%
{Die xPlanBox dient der Abbildung der Bauleit-, Regional- und Landschaftsplanung auf Basis von
XPlanGML. Der Vortrag stellt die auf Open-Source-Software basierte Lösung vor und zeigt die
Abbildung von Prozessen zur Verwaltung der Planwerke innerhalb einer PostgreSQL/PostGIS-Datenhaltung
sowie zur Validierung von XPlanGML"=Dateien auf. Die Publikation der Daten über deegree-WFS- und
WMS-Dienste sowie Einbindung in die Webclients OpenLayers und Geomajas werden im Rahmen des Vortrags
ebenfalls vorgestellt.}

\abstractElf{Marco Lechner}%
{Open-Source-Strategien im Notfallschutz}%
{Migration von proprietärer Software zu einer Open-Source-Entwicklung}%
{Um das System der Radioaktivitätsüberwachung weiterzuentwickeln, setzt das Bundesamt für
Strahlenschutz auf eine Open-Source-Strategie. Das Bundesamt für Strahlenschutz migriert derzeit sein IMIS-System von
einem proprietären zu einem komponentenorientierten Notfallinformationssystem. Das neue IMIS~3
integriert neben Eigenentwicklungen viele OSGeo-Projekte, wie PostGIS, GeoExt, OpenLayers~3,
Geoserver, GeoNetwork und MapfishPrint. Der Vortrag präsentiert GitHub-gehostete
Open-Source-Projekte des Bundesamts für Strahlenschutz, die in einem
Open-Source-gestützten Softwarestack eingebettet sind und mehrere bekannte
OSGeo-Projekte einsetzen.}


\abstractDreizehn{Alexander Lehner}%
{OSM und Freifunk}%
{Geodaten themenübergreifend nutzen}%
{%
Das Freifunk-Projekt erlebt zurzeit in Deutschland einen ähnlichen Aufschwung
wie OpenStreetMap in der Vergangenheit.
Durch das Anstecken eines Freifunk-Routers erweitert sich das Netzwerk um einen
weiteren Knoten.

Die aus dem
Heimgebrauch bekannte WLAN"=Funktechnik zieht einen
geografischen Bezug nach sich, weshalb hier zur Visualisierung von
Knotenpunkten auch die OSM"=Karte zum Einsatz kommt. Diese Knotenkarte ist jedoch
nicht statisch,
sondern wird live aktualisiert.
Dies erlaubt die Ansicht von WLAN-Verfügbarkeit und -Nutzung
in Echtzeit.
Hinzu liefert OSM Informationen zu den Standorten der Knoten und
deren näherer Umgebung. Ob die
Eck-Kneipe gut läuft und wohin die jungen Leute am Wochenende abends ausgehen.
All das verrät die
Freifunk"=Knotenkarte und vielleicht noch ein wenig mehr.}

\newtimeslot{16:30}
\abstractNeun{Till Adams}%
{SHOGun~2 -- das moderne Webmapping-Framework}%
{auf dem Weg zu einem Open-Source-Projekt}%
{SHOGun wird seit dem Jahr 2011 als Webmapping"=Framework entwickelt und hat sich mittlerweile in
  vielen Installationen, insbesondere bei Großkunden bewährt.  Es wird die Version SHOGun~2
  vorgestellt und anhand von Beispielen gezeigt, wie Kartenclient --
zusammengesetzt aus OpenLayers~3 und wahlweise GeoExt3 oder react.js -- mit den
Verwaltungsoberflächen zusammenspielen.  SHOGun~2 greift dabei bewährte Konzepte aus SHOGun auf,
wurde aber als komplette Neuentwicklung mit großem Fokus auf Modularität und Aktualität der
verwendeten Bibliotheken entwickelt.}


\abstractElf{Edgar Butwilowski}%
{Architektur moderner Geodatenportale}%
{}%
{Geodatenportale gewinnen immer mehr Nutzer. Dies ist jedoch keine Selbstverständlichkeit, sondern
das Ergebnis umfangreicher Verbesserungen bei Usability und Performanz. Heute werden interaktive und
zugängliche Geowebanwendungen verlangt, die den Nutzer auf kürzestem Wege zur gesuchten Information
führen. Doch was ist heute der State of the Art des Technologie-Stacks für die Entwicklung von
Geowebanwendungen?

Am Beispiel von geoportal.ch wird vorgestellt, wie fortschrittliche
Webtechnologien wie NodeJS, ExpressJS, Geoserver, Turf, AngularJS etc. zu einer Geoweb-Architektur
integriert werden.}

\abstractDreizehn{Alexander Matheisen}%
{ÖPNV-Mapping in OpenStreetMap}%
{Geht das nicht einfacher?}%
{Linienverläufe und Haltestellen des öffentlichen Verkehr werden schon seit vielen Jahren in
OpenStreetMap erfasst. In der letzten Zeit hat sich bei diesem Thema allerdings Ernüchterung
breitgemacht. Einstige Mapper wenden sich von diesem Themengebiet ab, neue Mapper werden von der
Komplexität des Mappings abgeschreckt und übrig bleibt ein Kern von wenigen Enthusiasten.
In diesem
Vortrag werden Erfahrungen und Probleme aus der Sicht eines langjährigen ÖPNV-Mappers dargestellt
und Anregungen für Vereinfachungen des ÖPNV"=Mappings gemacht, um mehr Mapper für dieses Thema zu
gewinnen.}

\newtimeslot{17:00}
\abstractNeun{Thomas Klein}%
{Das GeoMapFish-Web-GIS-Framework}%
{Neuerungen und Anwendungen der neuen Version 2.x}%
{Der Vortrag präsentiert das Open-Source-Web-GIS-Projekt GeoMapFish und gibt einen Überblick über
dessen Aufbau und Funktionen sowie einen kleinen Überblick der bereits damit umgesetzten
Fachanwendungen. Weiterhin werden bisherige und geplante Neuerungen, Funktionalitäten und
Erweiterungen vorgestellt, wie zum Beispiel die neue Editierfunktion in Abhängigkeit von
Nutzerrollen oder die Möglichkeit der QGIS-Server-Anbindung.}

\abstractElf{Stephan Herritsch}{Angular\,2-Geo-Apps mit YAGA}%
{}%
{Für moderne Geo-Anwendungen wird es immer wichtiger, dass sie sowohl auf
mobilen Geräten, als auch auf PCs einsetzbar sind. Benutzerfreudlichkeit
bedeutet nicht nur intuitive Bedienung, sondern auch Anpassung an den typischen
Look and Feel der Geräte.

\emph{YAGA leaflet-ng2} ist eine Integration von Leaflet in Angular2, das einen
modernen Ansatz für die technische Umsetzung durch Two-Way-Databinding
bietet. Diese ermöglicht eine einfache, modulare Entwicklungsweise mit klar
Trennung von Model und View.

Anhand von Beispielen, die sich an Anwender und Programmierer richten, sollen
die Vorteile der YAGA"=Komponenten im Allgemeinen und der Leaflet-Integration in
Angular 2 im Speziellen bei der plattformübergreifenden Arbeit mit Geodaten
darstellen.
}

%2017-03-23 18:00:00
\abstractDreizehn{Peter Barth}%
{OSM-Quiz}%
{Wie gut kennst du OSM?}%
{Das OSM-Quiz bietet als Fortsetzung des Events vom letztem Jahr wieder spannende Fragen zu
interessanten Fakten. Jeder ist herzlich eingeladen mitzuraten um sein Wissen im Umfeld von
OpenStreetMap und GIS zu testen und vielleicht auch etwas aufzufrischen.}

\newsmalltimeslot{18:00}
\abstractDreizehn{Daniel J H}{Open Source Routing Machine\label{bof-donnerstag}}{}{}
\vspace{-1\baselineskip}
\setabstract{}{FOSSGIS-Mitgliederversammlung}{}{}{dezentrot}{008}
