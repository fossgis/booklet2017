\section*{Goldsponsor}
%\begin{floatingfigure}[r]{0.5\textwidth}
%	\centering
%	\includegraphics[width=0.5\textwidth]{001_Wheregroup}
%	\vspace{1em}
%\end{floatingfigure}
\begin{center}
	\includegraphics[width=0.9\textwidth]{001_Wheregroup}
\end{center}
Die WhereGroup GmbH ist ein mittelständischer Dienst\-leister und Lösungsanbieter im Bereich 
webbasierter Geodateninfrastrukturen (GDI) und bietet Beratung, Konzeption, Entwicklung, 
Aufbau und Betrieb dynamischer Kartenanwendungen im Intra- und Internet. Da\-rüber 
hinaus gehört ein umfangreiches Schulungs- und Workshop-Pro\-gramm zum Portfolio.

Gegründet wurde das Unternehmen 2007 als eine Fusion der Unternehmen CCGIS GbR, 
Geo-Consortium GbR und KARTA.GO GmbH in Bonn. Bei der WhereGroup sind derzeit 25 
Angestellte aus unterschiedlichen Fachrichtungen am Hauptsitz in Bonn und der Niederlassung in Berlin tätig. 
Das Spektrum unserer Lösungen reicht von einfachen Lageskizzen über Geoportale und 
kartenbasierte Datenverwaltung bis hin zu hochverfügbaren Anwendungen für die freie 
Wirtschaft und die öffentliche Verwaltung. 

In unseren Projekten setzen wir auf die Standards bzw. Empfehlungen des Open Geospatial 
Consortiums (OGC), der INSPIRE-Richtlinie und der GDI-DE. Ihre Verwendung gewährleistet 
ein Maximum an Interoperabilität und Flexibilität unserer Lösungen. Die Einhaltung hoher 
Sicherheitsstandards ist für uns nicht zuletzt durch unsere Projekte mit Landes- und 
Bundesbehörden sowie Großkonzernen eine Selbstverständlichkeit.

Wir beraten absolut herstellerunabhängig und sind spe\-zialisiert auf die Weiter\-ent\-wicklung, professionelle Anwendung 
und Integration offener Standards und bewährter Open-Source-Technologien und 
freier Software. Dazu zäh\-len neben unseren Projekten Mapbender3, MetaDor2 und PostNAS u.\,a. 
GeoServer, MapServer, Map\-Proxy, Open Lay\-ers, PostGIS, QGIS und OpenStreetMap.
Zu unserer Überzeugung gehört, dass wir uns aktiv in der Geoinfor\-matik-Community engagieren. 
Es ist uns wichtig, an der Diskussion und Weiterentwicklung von verschiedensten Open\-Source-Lösungen mitzuwirken.

Die WhereGroup ist mit Hochschulen, Firmen und Verbänden bundesweit und international exzellent 
vernetzt. Wir verfügen über langjährige, persönliche Kontakte zu diversen Universitäten und 
Hochschulen im In- und Ausland, zum FOSSGIS e.V., zur Open Source Geospatial Foundation (OSGeo), 
zum Open Geospatial Consortium (OGC), sowie zu den Herstellern bzw. Maintainern der gängigsten 
Open-Source-Produkte im Geo-Bereich.
